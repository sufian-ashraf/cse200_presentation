% Real-world applications
\begin{frame}{Where Are Red-Black Trees Used?}
    \begin{center}
        \Large\textbf{Everywhere!}
    \end{center}
    
    \pause
    
    \begin{columns}
        \column{0.5\textwidth}
        \begin{itemize}
            \item<2-> \faIcon{linux} \textbf{Linux Kernel}\\
                \small Process scheduling
            \item<3-> \faIcon{java} \textbf{Java}\\
                \small TreeMap, TreeSet
            \item<4-> \faIcon{database} \textbf{Databases}\\
                \small Indexing structures
            \item<5-> \faIcon{folder} \textbf{File Systems}\\
                \small Directory organization
        \end{itemize}
        
        \column{0.5\textwidth}
        \only<6->{
            \begin{center}
                \begin{tikzpicture}
                    \node[draw, thick, rounded corners, fill=rbred!10, text width=5cm, align=center, minimum height=2cm] {
                        \Large\textbf{Every time you use these...}\\[0.3cm]
                        \large You're benefiting from\\
                        \textbf{Red-Black Trees!}
                    };
                \end{tikzpicture}
            \end{center}
        }
    \end{columns}
\end{frame}

% Conclusion
\begin{frame}{So There You Have It!}
    \begin{center}
        \Large\textbf{Red-Black Trees in a Nutshell}
    \end{center}
    
    \pause
    
    \begin{columns}
        \column{0.6\textwidth}
        \begin{itemize}
            \item<2-> Complex but incredibly powerful
            \item<3-> Tricky to implement
            \item<4-> Guaranteed $O(\log n)$ performance
            \item<5-> Used everywhere in computing
        \end{itemize}
        
        \column{0.4\textwidth}
        \only<6->{
            \begin{center}
                \begin{tikzpicture}[scale=0.6]
                    \node[blacknode] {11}
                        child {node[rednode] {2}
                            child {node[blacknode] {1}}
                            child {node[blacknode] {7}}
                        }
                        child {node[blacknode] {14}
                            child {node[nilnode] {}}
                            child {node[rednode] {15}}
                        };
                \end{tikzpicture}
            \end{center}
        }
    \end{columns}
    
    \pause[7]
    
    \vspace{0.5cm}
    
    \begin{center}
        \begin{alertblock}{Remember}
            The next time you're struggling with RBT implementation...\\[0.3cm]
            \large Even the \textbf{inventor} moved on to Left-Leaning Red-Black Trees! \faIcon{laugh}
        \end{alertblock}
    \end{center}
    
    \pause
    
    \begin{center}
        \Large\textbf{Thanks for listening!}\\[0.3cm]
        \large May your trees always stay balanced! \faIcon{tree}
    \end{center}
\end{frame}

% Questions slide
\begin{frame}
    \begin{center}
        \Huge\textbf{Any Questions?}\\[1cm]
        \Large\faIcon{comments}
    \end{center}
\end{frame}
