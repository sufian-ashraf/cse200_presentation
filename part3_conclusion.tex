% Rotation Complexity Proof Slide
\begin{frame}{Why Rotations Work: The Magic Behind Balance}
    \begin{center}
        \Large\textbf{Rotation Complexity: $O(1)$ Operations}
    \end{center}
    
    \pause
    
    \begin{columns}
        \column{0.5\textwidth}
        \textbf{Left Rotation:}
        \begin{itemize}
            \item<2-> Restructures tree locally
            \item<3-> Preserves binary search property
            \item<4-> Height changes by at most 1
        \end{itemize}
        
        \column{0.5\textwidth}
        \textbf{Right Rotation:}
        \begin{itemize}
            \item<2-> Mirror of left rotation
            \item<3-> Same time complexity
            \item<4-> Maintains red-black properties
        \end{itemize}
    \end{columns}
    
    \pause[5]
    
    \begin{alertblock}{Key Insight}
        Rotations are \textcolor{rbred}{\textbf{$O(1)$}} because they only modify a \textcolor{rbred}{\textbf{constant number of pointers}}!

        No tree traversal required.
    \end{alertblock}
\end{frame}

% Mathematical Proof Slide
\begin{frame}{Why Red-Black Trees Work: The Math Behind It}
    \begin{center}
        \Large\textbf{Height Proof: Why $O(\log n)$?}
    \end{center}
    
    \pause
    
    \begin{block}{Key Insight}
        A red-black tree with $n$ nodes has height $\leq {\color{rbred}\mathbf{2\log_2(n+1)}}$
    \end{block}
    
    \pause
    
    \begin{columns}[T]
        \column{0.45\textwidth}
        \textbf{Proof Sketch:}
        \begin{enumerate}
            \item<1-> Every path has \textcolor{rbred}{\textbf{same black-height}}
            \item<2-> \textcolor{rbred}{\textbf{No red-red}}  children
            \item<3-> $\geq$ 50\% nodes on path are \textcolor{rbred}{\textbf{black}}
            \item<4-> Height $\leq 2 \times$ black-height
        \end{enumerate}
        
        \column{0.45\textwidth}
        \only<5->{
            \begin{center}
                \begin{tikzpicture}[scale=0.45]
                    % Root node
                    \node[circle, draw=rbblack, fill=rbblack!15, text=rbblack, minimum size=6mm, font=\tiny\bfseries] (root) at (0, 0) {11};
                    % Level 1 nodes
                    \node[circle, draw=rbred, fill=rbred!20, text=rbred, minimum size=6mm, font=\tiny\bfseries] (left1) at (-3, -1.8) {2};
                    \node[circle, draw=rbblack, fill=rbblack!15, text=rbblack, minimum size=6mm, font=\tiny\bfseries] (right1) at (3, -1.8) {14};
                    % Level 2 nodes left subtree
                    \node[circle, draw=rbblack, fill=rbblack!15, text=rbblack, minimum size=6mm, font=\tiny\bfseries] (ll2) at (-4.5, -3.6) {1};
                    \node[circle, draw=rbblack, fill=rbblack!15, text=rbblack, minimum size=6mm, font=\tiny\bfseries] (lr2) at (-1.5, -3.6) {7};
                    % Level 2 nodes right subtree
                    \node[circle, draw=rbgray!90, fill=rbgray!60, text=white, minimum size=5mm, font=\tiny\bfseries] (rl2) at (1.5, -3.6) {N};
                    \node[circle, draw=rbred, fill=rbred!20, text=rbred, minimum size=6mm, font=\tiny\bfseries] (rr2) at (4.5, -3.6) {15};
                    % Level 3 nodes from node 7
                    \node[circle, draw=rbred, fill=rbred!20, text=rbred, minimum size=6mm, font=\tiny\bfseries] (lrl3) at (-2.5, -5.4) {5};
                    \node[circle, draw=rbgray!90, fill=rbgray!60, text=white, minimum size=5mm, font=\tiny\bfseries] (lrr3) at (-0.5, -5.4) {N};
                    % Edges
                    \draw (root) -- (left1);
                    \draw (root) -- (right1);
                    \draw (left1) -- (ll2);
                    \draw (left1) -- (lr2);
                    \draw (right1) -- (rl2);
                    \draw (right1) -- (rr2);
                    \draw (lr2) -- (lrl3);
                    \draw (lr2) -- (lrr3);
                \end{tikzpicture}
            \end{center}
            \small Black height = 2, Total height = 4
        }
    \end{columns}
    
    \pause[5]
    
\end{frame}

% Space Complexity Proof Slide  
\begin{frame}{Why Red-Black Trees Are Space Efficient}
    \begin{center}
        \Large\textbf{Space Complexity: $O(n)$ Memory Usage}
    \end{center}
    
    \pause
    
    \begin{columns}[T]
        \column{0.5\textwidth}
        {\textbf{Memory Requirements:}}
        \begin{itemize}
            \item<2-> Each node stores: \textcolor{badred}{\textbf{key + color + 2 pointers}}
            \item<3-> Total: $O(n)$ space
            \item<4-> No extra storage for balance info
        \end{itemize}
        
        \column{0.5\textwidth}
        {\textbf{Comparison:}}
        \begin{itemize}
            \item<2-> AVL trees: height field per node
            \item<3-> B-trees: multiple keys per node
            \item<4-> RBTs: minimal overhead
        \end{itemize}
    \end{columns}
    
    \pause[5]
    \vspace{0.1cm}
    \begin{alertblock}{Key Insight}
        RBTs achieve \textcolor{rbred}{\textbf{$O(n)$ space}} with \textcolor{rbred}{\textbf{1 bit}} per node (color field).
    \end{alertblock}
\end{frame}

% Modern Evolution Slide
\begin{frame}{The Evolution: From Red-Black to Modern Trees}
    \begin{center}
        \Large\textbf{50+ Years of Tree Balancing Innovation}
    \end{center}
    
    \pause
    
    \begin{columns}[T]
        \column{0.55\textwidth}
        \begin{itemize}
            \item<2-> \textbf{1972:} Red-Black Trees (Guibas \& Sedgewick) \cite{guibas1972}
            \item<3-> \textbf{2008:} Left-Leaning Red-Black Trees (Sedgewick) \cite{sedgewick2008}
            \item<4-> \textbf{2013:} WAVL Trees (Bronson et al.) \cite{bronson2013}
        \end{itemize}
        
        \column{0.35\textwidth}
        \only<5->{
            \begin{block}{What to do with these trees?}
                ``Trees evolving faster than my code!'' - Some sad developer
            \end{block}
        }
    \end{columns}
    
    \pause[6]
    
    \textcolor{rbred}{\textbf{Note:}} Even Sedgewick evolved the design—Left-Leaning RBTs reduce implementation complexity.
\end{frame}

% Advanced Applications
\begin{frame}{Beyond the Basics: Advanced Applications}
    \begin{center}
        \Large\textbf{Red-Black Trees in Modern Tech}
    \end{center}
    
    \pause
    
    \begin{columns}[T]
        \column{0.45\textwidth}
        \textbf{Traditional Uses:}
        \begin{itemize}
            \item<2-> \textcolor{rbred}{\textbf{Scheduling}}: CPU, process queues
            \item<3-> \textcolor{rbred}{\textbf{Memory}}: Allocation, defragmentation
            \item<4-> \textcolor{rbred}{\textbf{Routing}}: Network tables, prefixes
            \item<5-> \textcolor{rbred}{\textbf{Graphics}}: Spatial indexing, trees
        \end{itemize}
        
        \column{0.45\textwidth}
        \pause[6]
        \textbf{Modern Applications:}
        \begin{itemize}
            \item<7-> \textcolor{rbred}{\textbf{ML indexing}}, \textcolor{rbred}{\textbf{cloud storage}}, \textcolor{rbred}{\textbf{blockchain pathfinding}}
            \item<8-> Trusted by billions of devices daily
        \end{itemize}
    \end{columns}
    
\end{frame}

% Performance Comparison
\begin{frame}{RBT vs Other Trees}
    \begin{center}
        \Large\textbf{Why Choose Red-Black Trees?}
    \end{center}
    
    \pause
    
    \begin{table}
        \centering
        \begin{tabular}{|l|c|c|c|c|}
            \hline
            \textbf{Tree Type} & \textbf{Search} & \textbf{Insert} & \textbf{Delete} & \textbf{Space} \\
            \hline
            Red-Black & $O(\log n)$ & $O(\log n)$ & $O(\log n)$ & $O(n)$ \\
            AVL & $O(\log n)$ & $O(\log n)$ & $O(\log n)$ & $O(n)$ \\
            B-Tree & $O(\log n)$ & $O(\log n)$ & $O(\log n)$ & $O(n)$ \\
            \hline
        \end{tabular}
    \end{table}
    
    \pause
    
    \begin{columns}
        \column{0.45\textwidth}
        \begin{block}{Red-Black Advantages}
            \begin{itemize}
                \item<2-> \textcolor{rbred}{\textbf{Fewer rotations}} than AVL (faster inserts)
                \item<3-> \textcolor{rbred}{\textbf{Simpler code}} than B-trees
                \item<4-> \textcolor{rbred}{\textbf{Guaranteed}} $O(\log n)$ all operations
            \end{itemize}
        \end{block}
        \column{0.45\textwidth}
        \begin{center}
            \begin{block}{The Trade-off}
                \textit{Balanced performance across all operations}—optimal for general-purpose use.
            \end{block}
        \end{center}
    \end{columns}
    
\end{frame}

% Industry Secrets
\begin{frame}{Industry Secrets: What Big Tech Uses}
    \begin{center}
        \Large\textbf{The Real Story Behind the Scenes}
    \end{center}
    
    \pause
    
    \begin{columns}
        \column{0.55\textwidth}
        \textbf{Tech Giants \& RBTs:}
        \begin{itemize}
            \item<2-> \textcolor{rbred}{\textbf{Google}}: MapReduce distributed sorting
            \item<3-> \textcolor{rbred}{\textbf{Facebook}}: Feed ranking systems
            \item<4-> \textcolor{rbred}{\textbf{Amazon}}: Catalog indexing
            \item<5-> \textcolor{rbred}{\textbf{Netflix}}: Content routing
            \item<6-> \textcolor{rbred}{\textbf{Linux}}: Process scheduling
        \end{itemize}
        
        \column{0.35\textwidth}
        \only<7->{
            \begin{alertblock}{Production Strategy}
                \textcolor{rbred}{\textbf{Hybrid approaches}}: RBTs for small-medium datasets, B-trees for disk-based storage.
            \end{alertblock}
        }
    \end{columns}
    
    \pause[8]
    
\end{frame}

% Future Directions
\begin{frame}{The Future: Where Are Trees Heading?}
    \begin{center}
        \Large\textbf{Next Generation Data Structures}
    \end{center}
    
    \pause

    \begin{columns}
        \column{0.45\textwidth}
        \textbf{Emerging Trends:}
        \begin{itemize}
            \item<2-> {\textbf{Persistent Trees}}: Functional correctness guarantees
            \item<3-> {\textbf{Lock-free RBTs}}: Concurrent access patterns
            \item<4->{\textbf{GPU data structures}}: Massive parallelization
            \item<5-> {\textbf{Learned indices}}: ML-augmented trees
        \end{itemize}
        
        \column{0.45\textwidth}
        \only<6->{
            \begin{block}{Next Frontier}
                Adaptive structures: learning access patterns to optimize tree shape dynamically.
            \end{block}
        }
    \end{columns}
    
\end{frame}


% Final Conclusion
\begin{frame}{What Makes RBTs Essential}
    \begin{center}
        \Large\textbf{Why Red-Black Trees Matter}
    \end{center}
    
    \pause
    
    \begin{columns}
        \column{0.5\textwidth}
        \begin{itemize}
            \item<2-> \textcolor{badred}{\textbf{$O(\log n)$ guaranteed}}
            \item<3-> {\textbf{Practical efficiency}}—fewer rotations
            \item<4-> {\textbf{Mission-critical systems}}—everywhere
        \end{itemize}
        
        \column{0.5\textwidth}
        \only<5->{
            \begin{block}{The Foundation}
                35+ years of proven reliability in production systems.
            \end{block}
        }
    \end{columns}
    
    \pause[6]
    
\end{frame}

% Famous Words Slide
\begin{frame}{Words to Live By}
    \begin{center}
        \vspace{0.3cm}
        \begin{tikzpicture}
            \node[draw, thick, rounded corners=8pt, fill=rbred!10, text width=9cm, align=center, inner sep=0.5cm] {
                \Large\textit{``We know exactly how to balance a tree. We just haven't been outside to see one.''}\\[0.4cm]
                \normalsize --- Unknown
            };
        \end{tikzpicture}
    \end{center}
\end{frame}

\begin{frame}
    \begin{center}
        \Huge\textbf{Thanks for listening!}\\[0.5cm]
    \end{center}
\end{frame}
% % Questions slide
% \begin{frame}
%     \begin{center}
%         \Huge\textbf{Any Questions?}\\[1cm]
%         \Large\faIcon{comments}
%     \end{center}
% \end{frame}
