% \documentclass[aspectratio=169]{beamer}
% \usetheme{moloch}

% % Required packages
% \usepackage{tikz}
% \usepackage{amsmath,amssymb}
% \usepackage{graphicx}
% \usepackage{array}
% \usepackage{booktabs}
% \usepackage{biblatex}
% \addbibresource{references.bib}

% % TikZ libraries
% \usetikzlibrary{trees,positioning,shapes}

% % Define colors for red-black tree nodes
% \definecolor{rbred}{RGB}{220,20,60}
% \definecolor{rbblack}{RGB}{0,0,0}

% % Define TikZ node styles for red-black tree
% \tikzset{
%     blacknode/.style={circle,draw=rbblack,fill=rbblack,text=white,minimum size=8mm,font=\small},
%     rednode/.style={circle,draw=rbred,fill=rbred,text=white,minimum size=8mm,font=\small},
%     nilnode/.style={circle,draw=gray,fill=lightgray,minimum size=4mm,font=\tiny}
% }

% Title information
% \title{Red-Black Trees: The Unsung Heroes}
% \subtitle{Conclusion and Modern Perspectives}
% \author{B. M. Sufian Ashraf}
% \date{February 22, 2026}

% \begin{document}

% Title slide
\begin{frame}
	\titlepage
\end{frame}

% Rotation Complexity Proof Slide
\begin{frame}{Why Rotations Work: The Magic Behind Balance}
	\begin{center}
		\Large\textbf{Rotation Complexity: $O(1)$ Operations}
	\end{center}

	\pause

	\begin{columns}
		\tiny
		\column{0.5\textwidth}
		\textbf{Left Rotation:}
		\begin{itemize}
			\item<2-> Restructures tree locally
			\item<3-> Preserves binary search property
			\item<4-> Height changes by at most 1
		\end{itemize}

		\column{0.5\textwidth}
		\textbf{Right Rotation:}
		\begin{itemize}
			\item<2-> Mirror of left rotation
			\item<3-> Same time complexity
			\item<4-> Maintains red-black properties
		\end{itemize}
	\end{columns}

	\pause[5]

	\begin{alertblock}{Key Insight}
		Rotations are $O(1)$ because they only change a constant number of pointers!
		No tree traversal needed - just pointer gymnastics!
	\end{alertblock}

	\only<6->{
		\begin{center}
			\includegraphics[width=0.2\textwidth]{memes/pointer_magic.jpg}
		\end{center}
	}
\end{frame}

% Mathematical Proof Slide
\begin{frame}{Why Red-Black Trees Work: The Math Behind It}
	\begin{center}
		\Large\textbf{Height Proof: Why $O(\log n)$?}
	\end{center}

	\pause

	\begin{block}{Key Insight}
		A red-black tree with $n$ internal nodes has height $\leq 2\log_2(n+1)$
	\end{block}

	\pause

	\begin{columns}[T]
		\column{0.45\textwidth}
		\textbf{Proof Sketch:}
		\begin{enumerate}
			\item<1-> Every path from root to leaf has same number of black nodes
			\item<2-> Red nodes can't have red children
			\item<3-> At least half nodes on any path are black
			\item<4-> Height $\leq 2 \times$ black-height
		\end{enumerate}

		\column{0.45\textwidth}
		\only<5->{
			\begin{center}
				\begin{tikzpicture}[scale=0.6,
						level 1/.style={sibling distance=35mm},
						level 2/.style={sibling distance=20mm},
						level 3/.style={sibling distance=15mm}]
					\node[blacknode] {11}
					child {node[rednode] {2}
							child {node[blacknode] {1}}
							child {node[blacknode] {7}
									child {node[rednode] {5}}
									child {node[nilnode] {}}
								}
						}
					child {node[blacknode] {14}
							child {node[nilnode] {}}
							child {node[rednode] {15}}
						};
				\end{tikzpicture}
			\end{center}
			\small Black height = 2, Total height = 4
		}
	\end{columns}

	\pause[5]

\end{frame}

% Space Complexity Proof Slide  
\begin{frame}{Why Red-Black Trees Are Space Efficient}
	\begin{center}
		\Large\textbf{Space Complexity: $O(n)$ Memory Usage}
	\end{center}

	\pause

	\begin{columns}
		\column{0.5\textwidth}
		\textbf{Memory Requirements:}
		\begin{itemize}
			\item<2-> Each node stores: key, color and 2 pointers
			\item<3-> Total: $O(n)$ space
			\item<4-> No extra storage for balance info
		\end{itemize}

		\column{0.5\textwidth}
		\textbf{Comparison:}
		\begin{itemize}
			\item<2-> AVL trees: height field per node
			\item<3-> B-trees: multiple keys per node
			\item<4-> RBTs: minimal overhead \textbf{[squirrel-level efficient]}
		\end{itemize}
	\end{columns}

	\pause[5]

	\begin{alertblock}{Key Insight}
		Red-black trees achieve $O(n)$ space with just 1 extra bit per node (the color)!

		That's the definition of space-efficient data structures!
	\end{alertblock}
\end{frame}

% Modern Evolution Slide
\begin{frame}{The Evolution: From Red-Black to Modern Trees}
	\begin{center}
		\Large\textbf{50+ Years of Tree Balancing Innovation}
	\end{center}

	\pause

	\begin{columns}[T]
		\column{0.55\textwidth}
		\begin{itemize}
			\item<2-> \textbf{1972:} Red-Black Trees (Guibas \& Sedgewick) \cite{guibas1972}
			\item<3-> \textbf{2008:} Left-Leaning Red-Black Trees (Sedgewick) \cite{sedgewick2008}
			\item<4-> \textbf{2013:} WAVL Trees (Bronson et al.) \cite{bronson2013}
		\end{itemize}

		\column{0.35\textwidth}
		\only<5->{
			\begin{block}{What to do with these trees?}
				``Yikes! Trees evolving faster than my code.'' - Some sad developer
			\end{block}
		}
	\end{columns}

	\pause[6]

	\textbf{Fun fact:} Robert Sedgewick (co-inventor of RBT) later said: \textit{"I prefer left-leaning red-black trees now - they're simpler!"}
\end{frame}

% Advanced Applications
\begin{frame}{Beyond the Basics: Advanced Applications}
	\begin{center}
		\Large\textbf{Red-Black Trees in Modern Tech}
	\end{center}

	\pause

	\begin{columns}[T]
		\column{0.45\textwidth}
		\textbf{Traditional Uses:}
		\begin{itemize}
			\item<2-> CPU scheduling algorithms
			\item<3-> Memory management
			\item<4-> Network routing tables
			\item<5-> Game engines (spatial indexing)
		\end{itemize}

		\column{0.45\textwidth}
		\pause[6]
		\textbf{Modern Applications:}
		\begin{itemize}
			\item<7-> ML indexing, cloud storage, blockchain, AI pathfinding!
			\item<8-> RBTs: Keanu Reeves of data structures - always reliable!
		\end{itemize}
	\end{columns}

	\vspace{0.2cm}

	\begin{columns}[T]
		\column{0.45\textwidth}
		\only<7->{
			\begin{center}
				\includegraphics[width=0.5\columnwidth]{memes/rbt_world_domination_average_vs_gigachad.jpg}
			\end{center}
		}

		\column{0.45\textwidth}
		\only<8->{
			\begin{center}
				\includegraphics[width=0.7\columnwidth]{memes/whoa_keanu_reaves.jpg}
			\end{center}
		}
	\end{columns}


\end{frame}

% Performance Comparison
\begin{frame}{The Ultimate Showdown: RBT vs Other Trees}
	\begin{center}
		\Large\textbf{Why Choose Red-Black Trees?}
	\end{center}

	\pause

	\begin{table}
		\centering
		\begin{tabular}{|l|c|c|c|c|}
			\hline
			\textbf{Tree Type} & \textbf{Search} & \textbf{Insert} & \textbf{Delete} & \textbf{Space} \\
			\hline
			Red-Black          & $O(\log n)$     & $O(\log n)$     & $O(\log n)$     & $O(n)$         \\
			AVL                & $O(\log n)$     & $O(\log n)$     & $O(\log n)$     & $O(n)$         \\
			B-Tree             & $O(\log n)$     & $O(\log n)$     & $O(\log n)$     & $O(n)$         \\
			\hline
		\end{tabular}
	\end{table}

\pause

\end{frame}

% Red-Black Advantages with Memes
\begin{frame}{Red-Black Advantages: The Meme Edition}
	\begin{center}
		\textbf{Why Red-Black Trees Are the Perfect Choice}
	\end{center}
	
	\pause
	
	\begin{columns}[t]
		\column{0.28\textwidth}
		\only<2->{
			\begin{center}
				\includegraphics[width=0.9\columnwidth]{memes/goldilocks_is_this_a_pigeon.jpg}
			\end{center}
			\vspace{0.05cm}
			\footnotesize
			\textbf{The Goldilocks Solution}\\
			Not too fast, not too slow, just right!
		}
		
		\column{0.28\textwidth}
		\only<3->{
			\begin{center}
				\includegraphics[width=0.9\columnwidth]{memes/mic_drop_obama.jpg}
			\end{center}
			\vspace{0.05cm}
			\footnotesize
			\textbf{Rock-Solid Guarantees}\\
			Faster insertions than AVL, simpler than B-trees!
		}
		
		\column{0.28\textwidth}
		\only<4->{
			\begin{center}
				\includegraphics[width=0.9\columnwidth]{memes/balanced_coffee_distracted_boyfriend.jpg}
			\end{center}
			\vspace{0.05cm}
			\footnotesize
			\textbf{Perfect Balance}\\
			Like coffee - balanced and reliable!
		}
	\end{columns}
\end{frame}

% Industry Secrets
\begin{frame}{Industry Secrets: What Big Tech Uses}
	\begin{center}
		\Large\textbf{The Real Story Behind the Scenes}
	\end{center}

	\pause

	\begin{columns}
		\column{0.55\textwidth}
		\textbf{Tech Giants \& RBTs:}
		\begin{itemize}
			\item<2-> \textbf{Google:} Uses RBTs in MapReduce
			\item<3-> \textbf{Facebook:} News feed ranking
			\item<4-> \textbf{Amazon:} Product recommendations
			\item<5-> \textbf{Netflix:} Content delivery networks
			\item<6-> \textbf{Microsoft:} Windows kernel
		\end{itemize}

        \pause
		\column{0.35\textwidth}
		\textbf{Secret Sauce}\\
		Many companies use hybrid approaches - RBTs for small datasets, B-trees for large ones. It's complicated... but mostly red-black trees!
	\end{columns}

	\vspace{0.2cm}

	\only<8->{
		\begin{center}
			\includegraphics[width=0.4\textwidth,height=0.2\textwidth]{memes/conspiracy.jpg}
		\end{center}
	}

	\pause[8]

\end{frame}

% Future Directions
\begin{frame}{The Future: Where Are Trees Heading?}
	\begin{center}
		\Large\textbf{Next Generation Data Structures}
	\end{center}

	\pause

	\begin{columns}
		\column{0.45\textwidth}
		\textbf{Emerging Trends:}
		\begin{itemize}
			\item<2-> \textbf{Persistent Trees} (functional programming)
			\item<3-> \textbf{Concurrent RBTs} (multi-threading)
			\item<4-> \textbf{GPU-accelerated} trees
			\item<5-> \textbf{Quantum-inspired data structures - because we're not sure what they do, but they sound cool!}
		\end{itemize}

		\column{0.45\textwidth}
		\only<6->{
			\begin{center}
				\includegraphics[width=.9\textwidth,height=.9\textwidth]{memes/quantum_computing.jpg}
			\end{center}
		}
	\end{columns}
\end{frame}

% Final Conclusion with Memes
\begin{frame}{So There You Have It!}
	\begin{center}
		\Large\textbf{Red-Black Trees: The Unsung Heroes}
	\end{center}

	\pause

	\begin{columns}
		\column{0.5\textwidth}
		\begin{itemize}
			\item<2-> Mathematical elegance
			\item<3-> Practical performance
			\item<4-> Everywhere in computing
		\end{itemize}

		\column{0.5\textwidth}
		\only<5->{
			\begin{block}{Hero Status}
				Red-Black Trees: Not all heroes wear capes!
			\end{block}
		}
	\end{columns}

	\pause[6]

	\begin{alertblock}{Remember}
		The next time your code runs in $O(\log n)$ time...\\[0.3cm]
		\large Thank a red-black tree! $\heartsuit$
	\end{alertblock}
\end{frame}

% Thanks Slide
\begin{frame}
	\begin{center}
		\Huge\textbf{Thanks for listening!}\\[0.5cm]
        \pause
		\Large One Last thing before we go...\\[0.3cm]
        \pause
		\includegraphics[width=0.5\textwidth,height=0.3\textwidth,keepaspectratio]{memes/final_it_was_me_dio.jpg}
	\end{center}
\end{frame}

% \end{document}
