% ============================================================
% part2b_deletion.tex  –  Red-Black Tree: Deletion
% ============================================================

% ── Deletion intro ─────────────────────────────────────────
\begin{frame}{Deletion: Even More Fun!}
    \begin{center}
        \Large\textbf{Deletion is even more\ldots\ interesting!} \faIcon{smile}
    \end{center}

    \pause

    \begin{columns}
        \column{0.5\textwidth}
        \begin{center}
            \textbf{Deleting a \textcolor{rbred}{RED} node}
        \end{center}
        \begin{itemize}
            \item No problem!
            \item Just remove it
            \item Properties still hold
        \end{itemize}

        \column{0.5\textwidth}
        \pause
        \begin{center}
            \textbf{Deleting a \textbf{BLACK} node}
        \end{center}
        \begin{itemize}
            \item Oh boy\ldots
            \item Black height changes!
            \item Need ``double black'' fix
            \item Complex cases
        \end{itemize}
    \end{columns}

    \pause
    \vspace{0.4cm}
    \begin{center}
        \Large\textbf{Let's see both cases\ldots}
    \end{center}
\end{frame}

% ── Deletion Decision Flowchart (HORIZONTAL LAYOUT) ──────────
\begin{frame}{Deletion Decision Flowchart}
    \vspace{0.1cm}
    \begin{center}
    \begin{tikzpicture}[scale=0.8, every node/.style={transform shape},
        node distance = 0.6cm and 1.2cm,
        decision/.style = {diamond, draw=blue!60, fill=blue!12,
                           text width=3.5em, text centered,
                           inner sep=0pt, minimum height=2.3em,
                           font=\scriptsize},
        action/.style   = {rectangle, draw=green!60!black, fill=green!12,
                           text width=4.5em, text centered,
                           rounded corners=3pt, minimum height=1.8em,
                           font=\scriptsize},
        problem/.style  = {rectangle, draw=red!60, fill=red!12,
                           text width=4.5em, text centered,
                           rounded corners=3pt, minimum height=1.8em,
                           font=\scriptsize},
        arrow/.style    = {thick,->,>=stealth}]

        % Start node
        \node[decision] (del) {Delete\\Node};
        \node[decision, right=1.5cm of del] (isred) {Is node\\RED?};

        % TOP branch (YES - RED node - easy path)
        \node[action, above right=0.3cm and 1.2cm of isred] (remove) {Simply\\Remove};
        \node[action, right=0.8cm of remove] (done1) {\faIcon{check}\\Done!};

        % BOTTOM branch (NO - BLACK node - hard path)
        \node[problem, below right=0.3cm and 1.2cm of isred] (black) {BLACK\\Node};
        \node[decision, right=0.8cm of black] (sibling) {Check\\Sibling};
        \node[problem, right=0.8cm of sibling] (cases) {Apply\\4 Cases};
        \node[action, right=0.8cm of cases] (done2) {Rebalance\\\faIcon{check}};

        % Arrows
        \draw[arrow] (del) -- (isred);
        \draw[arrow] (isred) |- node[above,near start,font=\tiny]{Yes} (remove);
        \draw[arrow] (remove) -- (done1);
        \draw[arrow] (isred) |- node[below,near start,font=\tiny]{No} (black);
        \draw[arrow] (black) -- (sibling);
        \draw[arrow] (sibling) -- (cases);
        \draw[arrow] (cases) -- (done2);
    \end{tikzpicture}
    \end{center}

    \vspace{0.3cm}
    \begin{center}
        \textcolor{green!60!black}{\textbf{Top path (RED)}} = easy \quad
        \textcolor{red!60}{\textbf{Bottom path (BLACK)}} = complex
    \end{center}
\end{frame}

% ── Easy case: Deleting RED node ───────────────────────────
\begin{frame}{Case 1: Deleting a \textcolor{rbred}{RED} Node (Easy!)}
    \begin{center}
        \Large Delete \textbf{5} from our tree
    \end{center}

    \vspace{0.3cm}

    \begin{columns}
        \column{0.5\textwidth}
        \begin{center}
            \textbf{Before}\\[0.25cm]
            \begin{tikzpicture}[scale=0.8]
                \node[blacknode] {2}
                    child {node[blacknode] {1}}
                    child {node[blacknode] {4}
                        child {node[rednode] {3}}
                        child {node[rednode] {5}}
                    };
            \end{tikzpicture}
        \end{center}

        \column{0.5\textwidth}
        \pause
        \begin{center}
            \textbf{After}\\[0.25cm]
            \begin{tikzpicture}[scale=0.8]
                \node[blacknode] {2}
                    child {node[blacknode] {1}}
                    child {node[blacknode] {4}
                        child {node[rednode] {3}}
                        child[missing] {}
                    };
            \end{tikzpicture}
        \end{center}
    \end{columns}

    \vspace{0.3cm}
    \pause

    \begin{center}
        \textbf{Why It's Easy}
    \end{center}
    \begin{itemize}
        \item Node 5 is \textcolor{rbred}{RED} and a leaf
        \item Simply remove it — black-height unchanged on all paths
        \item All properties still satisfied!
    \end{itemize}

    \pause
    \begin{center}
        \Large\textcolor{green!60!black}{\faIcon{check}}\; Done! That was nice!
    \end{center}
\end{frame}

% ── Hard case: Deleting BLACK node ─────────────────────────
\begin{frame}{Case 2: Deleting a \textbf{BLACK} Node (Uh oh\ldots)}
    \begin{center}
        \Large Delete \textbf{1} from our tree
    \end{center}

    \vspace{0.2cm}

    \begin{columns}
        \column{0.33\textwidth}
        \begin{center}
            \textbf{Before}\\[0.15cm]
            \begin{tikzpicture}[scale=0.65]
                \node[blacknode] (n2) {2}
                    child {node[blacknode] (n1) {1}}
                    child {node[blacknode] (n4) {4}
                        child {node[rednode] {3}}
                        child[missing] {}
                    };
                \draw[red, very thick, dashed] (n1) circle (0.55cm);
            \end{tikzpicture}
        \end{center}

        \column{0.33\textwidth}
        \pause
        \begin{center}
            \textbf{After Delete}\\[0.15cm]
            \begin{tikzpicture}[scale=0.65, every node/.style={transform shape}]
                \node[blacknode] (root) {2}
                    child {node[draw=none,fill=none,text=gray!50] (left) {X}} % X is the double-black placeholder
                    child {node[blacknode] (right) {4}
                        child {node[rednode] (r3) {3}}
                        child[missing] {}
                    };
                
                % Stable Arrow for Short Path (Left)
                \draw[->,thick,red] (left.south) -- ++(0,-0.6) 
                    node[below, font=\tiny\bfseries] {Short};
                    
                % Stable Arrow for Long Path (Right)
                \draw[->,thick,green!50!black] (r3.south) -- ++(0,-0.6) 
                    node[below, font=\tiny\bfseries]{Long};
            \end{tikzpicture}\\[0.1cm]
            \fcolorbox{red!80!black}{red!15}{\textcolor{red!70!black}{\bfseries\footnotesize\faIcon{times}~IMBALANCED}}
        \end{center}

        \column{0.33\textwidth}
        \pause
        \begin{center}
            \textbf{After Fix}\\[0.15cm]
            \begin{tikzpicture}[scale=0.65, every node/.style={transform shape}]
                % The balanced result after rotation and recoloring
                \node[blacknode] (n3) {3}
                    child {node[blacknode] (n2) {2}}
                    child {node[blacknode] (n4) {4}};

                % Annotation for the rotation
                \draw[<->, bend left, thick, blue!60] (n3.north) to node[above, font=\tiny, text=blue!80!black] {Right Rotate} (n2.north west);
                
            \end{tikzpicture}\\[0.1cm]
            
            % Status Box
            \fcolorbox{green!60!black}{green!15}{%
                \textcolor{green!50!black}{\bfseries\footnotesize\faIcon{check-circle}~BALANCED}%
            }
            
            % Specific actions taken
            \vspace{0.2cm}
            \begin{itemize}
                \item {\tiny \faIcon{sync} \textbf{Rotate:} Right at 4, then 2}
                \item {\tiny \faIcon{paint-brush} \textbf{Recolor:} 3 $\to$ Black}
            \end{itemize}
        \end{center}
    \end{columns}

    \vspace{0.2cm}
    \pause
\end{frame}


\begin{frame}{Black Node Deletion: The 4 Cases}
    % Compact tree styling - shorter and wider
    \tikzset{
        every node/.style={transform shape, scale=0.35},
        level distance=12mm,
        sibling distance=8mm,
        highlight/.style={draw=blue, ultra thick, inner sep=1pt},
        ghost/.style={opacity=0.3}
    }

    \vspace{-0.2cm}

    % Using top-aligned columns for better vertical distribution
    \begin{columns}[t, onlytextwidth]
        % --- LEFT COLUMN ---
        \column{0.48\textwidth}
        \begin{block}{Case 1: Sibling RED}<1->
            \vspace{-0.25cm}
            \centering
            \begin{tikzpicture}[level distance=10mm, sibling distance=7mm]
                \node[blacknode] {P}
                    child {node[blacknode, ghost] {X}}
                    child {node[rednode, highlight] {S} 
                        child {node[blacknode] {L}}
                        child {node[blacknode] {R}}
                    };
            \end{tikzpicture}\\[-0.1cm]
            {\tiny Rotate \& Recolor \faIcon{sync}}
            \vspace{-0.25cm}
        \end{block}

        \vspace{0.1cm}

        \begin{block}{Case 3: Left RED}<3->
            \vspace{-0.25cm}
            \centering
            \begin{tikzpicture}[level distance=10mm, sibling distance=7mm]
                \node[blacknode] {P}
                    child {node[blacknode, ghost] {X}}
                    child {node[blacknode] {S}
                        child {node[rednode, highlight] {L}}
                        child {node[blacknode] {R}}
                    };
            \end{tikzpicture}\\[-0.1cm]
            {\tiny Right-rotate at S \faIcon{redo}}
            \vspace{-0.25cm}
        \end{block}

        % --- RIGHT COLUMN ---
        \column{0.48\textwidth}
        \begin{block}{Case 2: All BLACK}<2->
            \vspace{-0.25cm}
            \centering
            \begin{tikzpicture}[level distance=10mm, sibling distance=7mm]
                \node[blacknode] {P}
                    child {node[blacknode, ghost] {X}}
                    child {node[blacknode, highlight] {S}
                        child {node[blacknode] {L}}
                        child {node[blacknode] {R}}
                    };
            \end{tikzpicture}\\[-0.1cm]
            {\tiny Recolor S to RED \faIcon{paint-brush}}
            \vspace{-0.25cm}
        \end{block}

        \vspace{0.1cm}

        \begin{block}{Case 4: Right RED}<4->
            \vspace{-0.25cm}
            \centering
            \begin{tikzpicture}[level distance=10mm, sibling distance=7mm]
                \node[blacknode] {P}
                    child {node[blacknode, ghost] {X}}
                    child {node[blacknode] {S}
                        child {node[blacknode] {L}}
                        child {node[rednode, highlight] {R}}
                    };
            \end{tikzpicture}\\[-0.1cm]
            {\tiny Left-rotate at P \faIcon{undo}}
            \vspace{-0.25cm}
        \end{block}
    \end{columns}

    % Goal footer only appears on final step to save space initially
    \only<5->{
        \vspace{0.1cm}
        \begin{center}
            {\tiny\faIcon{lightbulb} Goal: Move RED to short path or recolor}
        \end{center}
    }
\end{frame}
