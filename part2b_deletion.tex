% ============================================================
% part2b_deletion.tex  –  Red-Black Tree: Deletion
% ============================================================

% ── Deletion intro ─────────────────────────────────────────
\begin{frame}{Deletion}
    \vspace{0.4cm}
    \begin{center}
        \Large\textbf{Deletion is even more\ldots\ interesting!}
    \end{center}

    \vspace{0.3cm}
    \uncover<2->{
    \begin{columns}[T]
        \column{0.5\textwidth}
        \begin{center}
            \large\textbf{Deleting a \textcolor{rbred}{RED} node}
        \end{center}
        \vspace{0.2cm}
        \begin{itemize}
            \item<3-> \large No problem!
            \item<4-> \large Just remove it
            \item<5-> \textcolor{goodgreen}{\large \textbf{Properties still hold}}
        \end{itemize}

        \column{0.5\textwidth}
        \begin{center}
            \large\textbf{Deleting a \textcolor{rbblack}{BLACK} node}
        \end{center}
        \vspace{0.2cm}
        \begin{itemize}
            \item<6-> \large Oh boy\ldots
            \item<7-> \large Black height changes!
            \item<8-> \textcolor{accent}{\large \textbf{Need ``double black'' fix}}
            \item<9-> \large Complex cases
        \end{itemize}
    \end{columns}
    }

    \uncover<10->{
    \vspace{0.5cm}
    \begin{center}
        \Large\textbf{Let's see both cases\ldots}
    \end{center}
    }
\end{frame}

% ── Deletion Decision Flowchart ──────────────────────────────
\begin{frame}{Deletion Decision Flowchart}
    \vspace{0.1cm}
    \begin{center}
    \begin{tikzpicture}[scale=0.82, every node/.style={transform shape},
        node distance = 0.5cm and 1.0cm,
        % Diamond: neutral decision node — amber theme
        decision/.style  = {diamond, draw=accent!90, fill=accent!18,
                            text=rbblack, text width=3.8em, text centered,
                            inner sep=0pt, minimum height=2.6em,
                            font=\small\bfseries},
        % Easy path nodes — green toned
        easynode/.style  = {rectangle, draw=goodgreen!80, fill=goodgreen!12,
                            text=goodgreen!70!black, text width=4.5em, text centered,
                            rounded corners=4pt, minimum height=2em,
                            font=\small\bfseries},
        % Easy path done — solid green
        easydone/.style  = {rectangle, draw=goodgreen, fill=goodgreen!25,
                            text=goodgreen!60!black, text width=4.5em, text centered,
                            rounded corners=4pt, minimum height=2em,
                            font=\small\bfseries},
        % Hard path nodes — red toned
        hardnode/.style  = {rectangle, draw=rbred!80, fill=rbred!12,
                            text=rbred!80!black, text width=4.5em, text centered,
                            rounded corners=4pt, minimum height=2em,
                            font=\small\bfseries},
        % Hard path done — muted but resolved
        harddone/.style  = {rectangle, draw=goodgreen!70, fill=goodgreen!15,
                            text=goodgreen!60!black, text width=4.5em, text centered,
                            rounded corners=4pt, minimum height=2em,
                            font=\small\bfseries},
        % Arrows per path
        arrowbase/.style = {thick,->,>=stealth},
        arroweasy/.style = {arrowbase, goodgreen!70!black},
        arrowhard/.style = {arrowbase, rbred!70!black},
        arrow/.style     = {arrowbase, rbblack!60}]

        % Start + decision
        \node[decision] (del)   {Delete\\Node};
        \node[decision, right=1.5cm of del] (isred) {Is node\\RED?};

        % TOP branch — easy (green)
        \node[easynode, above right=0.3cm and 1.2cm of isred] (remove) {Simply\\Remove};
        \node[easydone, right=0.9cm of remove] (done1) {\faIcon{check}\ Done!};

        % BOTTOM branch — hard (red)
        \node[hardnode, below right=0.3cm and 1.2cm of isred] (black)   {BLACK\\Node};
        \node[decision, right=0.9cm of black]   (sibling) {Check\\Sibling};
        \node[hardnode, right=0.9cm of sibling] (cases)   {Apply\\4 Cases};
        \node[harddone, right=0.9cm of cases]   (done2)   {Rebalance\\\faIcon{check}};

        % Arrows
        \draw[arrow]     (del)    -- (isred);
        \draw[arroweasy] (isred)  |- node[above, near start, font=\scriptsize\bfseries,
                                          text=goodgreen!70!black]{Yes} (remove);
        \draw[arroweasy] (remove) -- (done1);
        \draw[arrowhard] (isred)  |- node[below, near start, font=\scriptsize\bfseries,
                                          text=rbred!80!black]{No}  (black);
        \draw[arrowhard] (black)  -- (sibling);
        \draw[arrowhard] (sibling)-- (cases);
        \draw[arroweasy] (cases)  -- (done2);
    \end{tikzpicture}
    \end{center}

    \vspace{0.3cm}
    \begin{center}
        \large\textcolor{goodgreen!70!black}{\textbf{Top path (RED node)}} = straightforward \\[0.15cm]
        \large\textcolor{rbred!80!black}{\textbf{Bottom path (BLACK node)}} = complex
    \end{center}
\end{frame}
% ── Easy case: Deleting RED node – trees ───────────────────
\begin{frame}{Case 1: Deleting a \textcolor{rbred}{RED} Node}
    \begin{center}
        \Large Delete node \textbf{5} from the tree
    \end{center}
    \vspace{0.3cm}
    \begin{columns}[c]
        \column{0.5\textwidth}
        \begin{center}
            \large\textbf{Before}\\[0.4cm]
            \begin{tikzpicture}[scale=1.0, every node/.style={font=\small\bfseries}]
                \node[blacknode] (root) at ( 0,    0)    {2};
                \node[blacknode] (n1)   at (-1.5, -1.5)  {1};
                \node[blacknode] (n4)   at ( 1.5, -1.5)  {4};
                \node[rednode]   (n3)   at ( 0.7, -3.0)  {3};
                \node[rednode]   (n5)   at ( 2.3, -3.0)  {5};
                \draw[thick] (root)--(n1); \draw[thick] (root)--(n4);
                \draw[thick] (n4)--(n3);   \draw[thick] (n4)--(n5);
                \draw[rbred, very thick, dashed] (n5) circle (0.55cm);
            \end{tikzpicture}
        \end{center}

        \column{0.5\textwidth}
        \pause
        \begin{center}
            \large\textbf{After}\\[0.4cm]
            \begin{tikzpicture}[scale=1.0, every node/.style={font=\small\bfseries}]
                \node[blacknode] (root) at ( 0,    0)    {2};
                \node[blacknode] (n1)   at (-1.5, -1.5)  {1};
                \node[blacknode] (n4)   at ( 1.5, -1.5)  {4};
                \node[rednode]   (n3)   at ( 0.7, -3.0)  {3};
                \draw[thick] (root)--(n1); \draw[thick] (root)--(n4);
                \draw[thick] (n4)--(n3);
            \end{tikzpicture}
        \end{center}
    \end{columns}
    \vspace{0.3cm}
\end{frame}

\begin{frame}{Case 1: Deleting a \textcolor{rbred}{RED} Node}
    \vspace{1.8cm}
    \begin{center}
        \LARGE\textbf{Is it done?}\\[0.5cm]
        \uncover<2->{
            \Large\textcolor{badred}{Let's check the black height.}
        }
    \end{center}
\end{frame}


% % ── Easy case: Why It's Easy ────────────────────────────────
% \begin{frame}{Case 1: Why It's Easy}
%     \vspace{0.5cm}
%     \begin{itemize}
%         \item<1-> \large Node 5 is \textcolor{rbred}{\textbf{RED}} and a leaf
%         \item<2-> \large Simply remove it — black-height \textbf{unchanged} on all paths
%         \item<3-> \large All 5 Red-Black properties still satisfied!
%     \end{itemize}

%     \vspace{0.6cm}
%     \uncover<4->{
%     \begin{center}
%         \Large\textcolor{goodgreen}{\faIcon{check}}\enspace\textbf{Done!}
%     \end{center}
%     }
% \end{frame}

% ── Easy case: Black-height unchanged simulation ─────────────
\begin{frame}{Case 1: Black-Height Stays the Same}
    \begin{center}
        \large Every path still has \textbf{bc = 2} black nodes after removing 5
    \end{center}
    \vspace{0.3cm}
    \begin{columns}[c]
        \column{0.5\textwidth}
        \begin{center}
            \large\textbf{Before (with node 5)}\\[0.3cm]
            \begin{tikzpicture}[scale=1.0, every node/.style={font=\small\bfseries}]
                \node[blacknode] (root) at ( 0,    0)    {2};
                \node[blacknode] (n1)   at (-1.5, -1.5)  {1};
                \node[blacknode] (n4)   at ( 1.5, -1.5)  {4};
                \node[rednode]   (n3)   at ( 0.7, -3.0)  {3};
                \node[rednode]   (n5)   at ( 2.3, -3.0)  {5};
                \draw[thick] (root)--(n1); \draw[thick] (root)--(n4);
                \draw[thick] (n4)--(n3);   \draw[thick] (n4)--(n5);
                % Path 1: root -> 1
                \only<2->{
                    \draw[line width=2.5pt, orange!80!black] (root) -- (n1);
                    \node[below=0.1cm of n1, font=\scriptsize, text=orange!80!black] {\textbf{bc=2}};
                }
                % Path 2: root -> 4 -> 3
                \only<3->{
                    \draw[line width=2.5pt, goodgreen] (root) -- (n4) -- (n3);
                    \node[below=0.1cm of n3, font=\scriptsize, text=goodgreen] {\textbf{bc=2}};
                }
                % Path 3: root -> 4 -> 5
                \only<4->{
                    \draw[line width=2.5pt, accent] (root) -- (n4) -- (n5);
                    \node[below=0.1cm of n5, font=\scriptsize, text=accent] {\textbf{bc=2}};
                }
            \end{tikzpicture}
        \end{center}

        \column{0.5\textwidth}
        \uncover<5->{
        \begin{center}
            \large\textbf{After (node 5 removed)}\\[0.3cm]
            \begin{tikzpicture}[scale=1.0, every node/.style={font=\small\bfseries}]
                \node[blacknode] (root) at ( 0,    0)    {2};
                \node[blacknode] (n1)   at (-1.5, -1.5)  {1};
                \node[blacknode] (n4)   at ( 1.5, -1.5)  {4};
                \node[rednode]   (n3)   at ( 0.7, -3.0)  {3};
                \draw[thick] (root)--(n1); \draw[thick] (root)--(n4);
                \draw[thick] (n4)--(n3);
                % Path 1
                \only<6->{
                    \draw[line width=2.5pt, orange!80!black] (root) -- (n1);
                    \node[below=0.1cm of n1, font=\scriptsize, text=orange!80!black] {\textbf{bc=2}};
                }
                % Path 2
                \only<7->{
                    \draw[line width=2.5pt, goodgreen] (root) -- (n4) -- (n3);
                    \node[below=0.1cm of n3, font=\scriptsize, text=goodgreen] {\textbf{bc=2}};
                }
            \end{tikzpicture}
        \end{center}
        }
    \end{columns}
\end{frame}

% ── Hard case: Deleting BLACK node – Before ─────────────────
\begin{frame}{Case 2: Deleting a \textbf{BLACK} Node}
    \begin{center}
        \Large Delete node \textbf{1} from the tree
    \end{center}

    \vspace{0.4cm}

    \begin{center}
        \large\textbf{Before deletion:}\\[0.4cm]
        \begin{tikzpicture}[scale=1.0, every node/.style={font=\small\bfseries}]
            \node[blacknode] (n2) at ( 0,    0)    {2};
            \node[blacknode] (n1) at (-1.5, -1.5)  {1};
            \node[blacknode] (n4) at ( 1.5, -1.5)  {4};
            \node[rednode]   (n3) at ( 0.7, -3.0)  {3};
            \draw[thick] (n2)--(n1); \draw[thick] (n2)--(n4);
            \draw[thick] (n4)--(n3);
            \draw[rbred, very thick, dashed] (n1) circle (0.55cm);
        \end{tikzpicture}
    \end{center}
\end{frame}

% ── Hard case: After Delete ─────────────────────────────────
\begin{frame}{Case 2: After Deleting the BLACK Node}
    \begin{center}
        \Large The node is gone - but now we have a \textbf{problem}
    \end{center}

    \vspace{0.2cm}

    \begin{columns}[c]
        \column{0.55\textwidth}
        \begin{center}
        \begin{tikzpicture}[scale=1.0, every node/.style={font=\small\bfseries}]
            \node[blacknode] (root)  at ( 0,    0)    {2};
            \node[draw=rbblack!50, fill=rbblack!8, circle, dashed, minimum size=8mm,
                  font=\bfseries, text=rbblack] (left) at (-1.5, -1.5) {?};
            \node[blacknode] (right) at ( 1.5, -1.5)  {4};
            \node[rednode]   (r3)    at ( 0.7, -3.0)  {3};
            \draw[thick] (root)--(left); \draw[thick] (root)--(right);
            \draw[thick] (right)--(r3);
            \draw[->,thick,rbred] (left.south) -- ++(0,-0.5)
                node[below, xshift=-2mm, font=\small\bfseries] {Short};
            \draw[->,thick,goodgreen] (r3.south) -- ++(0,-0.5)
                node[below, font=\small\bfseries]{Long};
        \end{tikzpicture}
        \end{center}

        \column{0.45\textwidth}
        \begin{itemize}
            \pause\item \large Left path is now \textbf{shorter}
            \pause\item \large Black-height \textcolor{badred}{\textbf{violated!}}
            \pause\item \large We call this a\\[0.2cm]
                        \large\textcolor{rbred}{\textbf{``Double-Black''}} node
        \end{itemize}
    \end{columns}

    \vspace{0.1cm}
    \pause
    \begin{center}
        \large\textcolor{badred}{\enspace\textbf{IMBALANCED - must fix!}}
    \end{center}
\end{frame}

% ── Hard case: After Fix ─────────────────────────────────────
\begin{frame}{Case 2: The Fix-up}
    \begin{columns}[T]
        \column{0.45\textwidth}
        \vspace{0.6cm}
        \begin{itemize}
            \item<1-> \large \enspace\textbf{Rotate:} Right at 4,\\then left at 2
            \item<2-> \large \enspace\textbf{Recolor:} Node 3 $\to$ Black
            \item<6-> \large \textcolor{highlight}{\enspace \textbf{Tree is balanced!}}
        \end{itemize}

        \column{0.55\textwidth}
        \uncover<3->{
        \begin{center}
        \begin{tikzpicture}[scale=1.0, every node/.style={font=\small\bfseries}]
            \node[blacknode] (r3) at ( 0,    0)    {3};
            \node[blacknode] (n2) at (-1.2, -1.5)  {2};
            \node[blacknode] (n4) at ( 1.2, -1.5)  {4};
            \draw[thick] (r3)--(n2); \draw[thick] (r3)--(n4);
            \only<4->{
                \draw[line width=2.5pt, orange!80!black] (r3) -- (n2);
                \node[below=0.15cm of n2, font=\small\bfseries,
                      text=orange!80!black] {bc=2};
            }
            \only<5->{
                \draw[line width=2.5pt, green!60!black] (r3) -- (n4);
                \node[below=0.15cm of n4, font=\small\bfseries,
                      text=green!60!black] {bc=2};
            }
        \end{tikzpicture}
        \end{center}
        }
    \end{columns}

    \uncover<6->{
    \begin{center}
        \large\textcolor{goodgreen}{\enspace\textbf{All paths: bc\,=\,2 - Black-height restored!}}
    \end{center}
    }
\end{frame}

% ── Double-Black: what are the 4 cases? ─────────────────────
\begin{frame}{Fixing Double-Black: 4 Cases}
    \vspace{0.3cm}
    \begin{center}
        \Large When we have a \textcolor{rbred}{\textbf{Double-Black}} node,\\[0.3cm]
        \Large the fix depends on the \textbf{sibling's color and children}.
    \end{center}

    \vspace{0.6cm}

    \begin{center}
    {\large
        \textbf{P} = Parent \qquad
        \textbf{S} = Sibling \qquad
        \textbf{L / R} = S's children
    }\\[0.5cm]
    {\large
        \tikz[baseline=-0.5ex]\node[circle, draw=rbblack, fill=rbblack!10,
            text=rbblack, dashed, font=\bfseries\normalsize,
            minimum size=10mm, inner sep=0pt]{DB}; \enspace = Double-Black node
    }
    \end{center}

    \vspace{0.5cm}
    \pause
    \begin{center}
        \Large\textcolor{accent}{\enspace\textbf{4 cases - let's go through them one by one!}}
    \end{center}
\end{frame}

% ── Case 1 of 4: Sibling is RED ─────────────────────────────
\begin{frame}{Fix Case 1 of 4: Sibling is \textcolor{rbred}{RED}}
    \begin{center}
        \large\textcolor{badred}{\enspace\textbf{The Sibling S is RED}}
    \end{center}

    \vspace{0.05cm}

    \begin{columns}[T]
        \column{0.45\textwidth}
        \begin{center}
            \small\textbf{Before}\\[0.2cm]
            \begin{tikzpicture}[scale=0.75, every node/.style={font=\small\bfseries}]
                \node[blacknode]   (P)  at ( 0,    0)    {P};
                \node[circle,draw=rbblack!70,fill=rbblack!8,text=rbblack,dashed,
                      minimum size=9mm] (DB) at (-1.2, -1.4) {DB};
                \node[rednode,ultra thick] (S) at ( 1.2, -1.4) {S};
                \node[blacknode]   (L)  at ( 0.4, -2.8) {L};
                \node[blacknode]   (R)  at ( 2.0, -2.8) {R};
                \draw[thick] (P)--(DB); \draw[thick] (P)--(S);
                \draw[thick] (S)--(L); \draw[thick] (S)--(R);
            \end{tikzpicture}
        \end{center}

        \column{0.55\textwidth}
        \uncover<5->{
        \begin{center}
            \small\textbf{After}\\[0.2cm]
            \begin{tikzpicture}[scale=0.75, every node/.style={font=\small\bfseries}]
                \node[blacknode]   (S)  at ( 0,    0)    {S};
                \node[rednode]     (P)  at (-1.2, -1.4) {P};
                \node[blacknode]   (R)  at ( 1.2, -1.4) {R};
                \node[circle,draw=rbblack!70,fill=rbblack!8,text=rbblack,dashed,
                      minimum size=9mm] (DB) at (-2.0, -2.8) {DB};
                \node[blacknode]   (L)  at (-0.4, -2.8) {L};
                \draw[thick] (S)--(P); \draw[thick] (S)--(R);
                \draw[thick] (P)--(DB); \draw[thick] (P)--(L);
            \end{tikzpicture}\\[0.15cm]
            \textcolor{accent}{\textit{\textbf{Now apply Case 2, 3, or 4 to DB}}}
        \end{center}
        }
    \end{columns}

    \vspace{0.05cm}
    \begin{itemize}
        \item<2-> \large \enspace\textbf{Rotate} P to the left
        \item<3-> \large\enspace\textbf{Recolor:} S $\to$ Black, P $\to$ Red
        % \item<4-> \large\textcolor{accent}{\textit{This converts it into Case 2, 3, or 4}}
    \end{itemize}
\end{frame}

% ── Case 2 of 4: Sibling & children all BLACK ───────────────
\begin{frame}{Fix Case 2 of 4: Sibling \& Children All BLACK}
    \begin{center}
        \large\textcolor{badred}{\enspace\textbf{S and both children are BLACK}}
    \end{center}

    \vspace{0.05cm}

    \begin{columns}[T]
        \column{0.45\textwidth}
        \begin{center}
            \small\textbf{Before}\\[0.2cm]
            \begin{tikzpicture}[scale=0.75, every node/.style={font=\small\bfseries}]
                \node[blacknode]   (P)  at ( 0,    0)    {P};
                \node[circle,draw=rbblack!70,fill=rbblack!8,text=rbblack,dashed,
                      minimum size=9mm] (DB) at (-1.2, -1.4) {DB};
                \node[blacknode]   (S)  at ( 1.2, -1.4) {S};
                \node[blacknode]   (L)  at ( 0.4, -2.8) {L};
                \node[blacknode]   (R)  at ( 2.0, -2.8) {R};
                \draw[thick] (P)--(DB); \draw[thick] (P)--(S);
                \draw[thick] (S)--(L); \draw[thick] (S)--(R);
            \end{tikzpicture}
        \end{center}

        \column{0.55\textwidth}
        \uncover<5->{
        \begin{center}
            \small\textbf{After}\\[0.2cm]
            \begin{tikzpicture}[scale=0.75, every node/.style={font=\small\bfseries}]
                \node[circle,draw=rbblack!70,fill=rbblack!8,text=rbblack,dashed,
                      minimum size=9mm] (P) at ( 0,    0)    {P};
                \node[blacknode]   (X)  at (-1.2, -1.4) {X};
                \node[rednode]     (S)  at ( 1.2, -1.4) {S};
                \node[blacknode]   (L)  at ( 0.4, -2.8) {L};
                \node[blacknode]   (R)  at ( 2.0, -2.8) {R};
                \draw[thick] (P)--(X); \draw[thick] (P)--(S);
                \draw[thick] (S)--(L); \draw[thick] (S)--(R);
            \end{tikzpicture}\\[0.15cm]
            \textcolor{accent}{\textit{\textbf{DB pushed to P — continue fixing}}}
        \end{center}
        }
    \end{columns}

    \vspace{0.05cm}
    \begin{itemize}
        \item<2-> \large \enspace\textbf{Recolor} S $\to$ Red
        \item<3-> \large \enspace Push the Double-Black \textbf{up to P}
        % \item<4-> \large\textcolor{accent}{\textit{Repeat fix from P if P was also Black}}
    \end{itemize}
\end{frame}

% ── Case 3 of 4: Sibling's Left child is RED ────────────────
\begin{frame}{Fix Case 3 of 4: Sibling's \textcolor{rbred}{Left} Child is \textcolor{rbred}{RED}}
    \begin{center}
        \large\textcolor{badred}{\enspace\textbf{S is Black, S's Left child is RED}}
    \end{center}

    \vspace{0.05cm}

    \begin{columns}[T]
        \column{0.45\textwidth}
        \begin{center}
            \small\textbf{Before}\\[0.2cm]
            \begin{tikzpicture}[scale=0.75, every node/.style={font=\small\bfseries}]
                \node[blacknode]   (P)  at ( 0,    0)    {P};
                \node[circle,draw=rbblack!70,fill=rbblack!8,text=rbblack,dashed,
                      minimum size=9mm] (DB) at (-1.2, -1.4) {DB};
                \node[blacknode]   (S)  at ( 1.2, -1.4) {S};
                \node[rednode,ultra thick] (L) at ( 0.4, -2.8) {L};
                \node[blacknode]   (R)  at ( 2.0, -2.8) {R};
                \draw[thick] (P)--(DB); \draw[thick] (P)--(S);
                \draw[thick] (S)--(L); \draw[thick] (S)--(R);
            \end{tikzpicture}
        \end{center}

        \column{0.55\textwidth}
        \uncover<5->{
        \begin{center}
            \small\textbf{After}\\[0.2cm]
            \begin{tikzpicture}[scale=0.6, every node/.style={font=\small\bfseries},
                blacknode/.append style={minimum size=6mm},
                rednode/.append style={minimum size=6mm}]
                \node[blacknode]   (P)  at (-0.3,  0)    {P};
                \node[circle,draw=rbblack!70,fill=rbblack!8,text=rbblack,dashed,
                      minimum size=6mm,font=\tiny\bfseries] (DB) at (-1.4, -1.4) {DB};
                \node[blacknode]   (L)  at ( 0.8, -1.4) {L};
                \node[rednode]     (S)  at ( 1.5, -2.8) {S};
                \node[blacknode]   (R)  at ( 2.2, -4.2) {R};
                \draw[thick] (P)--(DB); \draw[thick] (P)--(L);
                \draw[thick] (L)--(S); \draw[thick] (S)--(R);
            \end{tikzpicture}\\[0.15cm]
            \textcolor{accent}{\textit{\textbf{Now proceed with Case 4}}}
        \end{center}
        }
    \end{columns}

    \vspace{0.05cm}
    \begin{itemize}
        \item<2-> \large \enspace\textbf{Right-rotate} at S, \& \textbf{Swap colors} of S and L
        % \item<3-> \large \enspace\textbf{Swap colors} of S and L
        % \item<4-> \large\textcolor{accent}{\textit{This transforms the situation into Case 4}}
    \end{itemize}
\end{frame}

% ── Case 4 of 4: Sibling's Right child is RED ───────────────
\begin{frame}{Fix Case 4 of 4: Sibling's \textcolor{rbred}{Right} Child is \textcolor{rbred}{RED}}
    \begin{center}
        \large\textcolor{badred}{\enspace\textbf{S is Black, S's Right child is RED}}
    \end{center}

    \vspace{0.05cm}

    \begin{columns}[T]
        \column{0.45\textwidth}
        \begin{center}
            \small\textbf{Before}\\[0.2cm]
            \begin{tikzpicture}[scale=0.75, every node/.style={font=\small\bfseries}]
                \node[blacknode]   (P)  at ( 0,    0)    {P};
                \node[circle,draw=rbblack!70,fill=rbblack!8,text=rbblack,dashed,
                      minimum size=9mm] (DB) at (-1.2, -1.4) {DB};
                \node[blacknode]   (S)  at ( 1.2, -1.4) {S};
                \node[blacknode]   (L)  at ( 0.4, -2.8) {L};
                \node[rednode,ultra thick] (R) at ( 2.0, -2.8) {R};
                \draw[thick] (P)--(DB); \draw[thick] (P)--(S);
                \draw[thick] (S)--(L); \draw[thick] (S)--(R);
            \end{tikzpicture}
        \end{center}

        \column{0.55\textwidth}
        \uncover<5->{
        \begin{center}
            \small\textbf{After}\\[0.2cm]
            \begin{tikzpicture}[scale=0.75, every node/.style={font=\small\bfseries}]
                \node[blacknode]   (S)  at ( 0,    0)    {S};
                \node[blacknode]   (P)  at (-1.2, -1.4) {P};
                \node[blacknode]   (R)  at ( 1.2, -1.4) {R};
                \node[blacknode]   (X)  at (-2.0, -2.8) {X};
                \node[blacknode]   (L)  at (-0.4, -2.8) {L};
                \draw[thick] (S)--(P); \draw[thick] (S)--(R);
                \draw[thick] (P)--(X); \draw[thick] (P)--(L);
            \end{tikzpicture}\\[0.15cm]
            \Large \textcolor{goodgreen}{\textit{\textbf{Double-Black fully resolved!}}}
        \end{center}
        }
    \end{columns}

    \vspace{0.05cm}
    \begin{itemize}
        \item<2-> \large \enspace\textbf{Left-rotate} at P
        \item<3-> \large \enspace\textbf{Recolor} R $\to$ Black
        % \item<4-> \large\textcolor{goodgreen}{\faIcon{check}\enspace\textbf{Double-Black fully resolved!}}
    \end{itemize}
\end{frame}

% ── Case 4 goal ─────────────────────────────────────────────
\begin{frame}{Summary}
    \vspace{0.5cm}
    \begin{center}
        \Large The cases form a \textcolor{highlight}{\textbf{chain}}:
    \end{center}

    \vspace{0.6cm}

    \begin{center}
    \begin{tikzpicture}[scale=1.0,
        node distance=1.6cm,
        box/.style={rectangle, draw=accent!80, fill=accent!15,
                    text=rbblack, font=\normalsize\bfseries,
                    rounded corners=5pt, minimum width=2.2cm, minimum height=1cm,
                    text centered},
        arr/.style={thick,->,>=stealth,accent!80}]
        \node[box] (c1) {Case 1};
        \node[box, right=of c1] (c23) {Case 2/3};
        \node[box, right=of c23] (c4) {Case 4};
        \node[rectangle, draw=goodgreen, fill=goodgreen!15, text=goodgreen,
              font=\normalsize\bfseries, rounded corners=5pt,
              minimum width=2.4cm, minimum height=1cm,
              text centered, right=of c4] (done) {\faIcon{check} Done!};

        \draw[arr] (c1) -- node[above,font=\small]{rotate} (c23);
        \draw[arr] (c23) -- node[above,font=\small]{transform} (c4);
        \draw[arr] (c4) -- node[above,font=\small]{rotate} (done);
    \end{tikzpicture}
    \end{center}

    \vspace{0.6cm}
    \pause
    \begin{center}
        \large\textcolor{accent}{\enspace\textbf{Goal:} Eventually reach Case 4 to fully eliminate Double-Black}
    \end{center}
    \uncover<3->{
    \vspace{0.2cm}
    \begin{center}
        \large\textcolor{gray}{\textit{Case 2 may propagate upward; Cases 1 \& 3 always lead to Case 4}}
    \end{center}
    }
\end{frame}

% ── Confused Meme Slide ─────────────────────────────────────
\begin{frame}{Too Many Cases?}
    \begin{columns}[c]
        \column{0.46\textwidth}
        \begin{center}
            \includegraphics[width=0.88\textwidth]{images/confused_cat.jpg}
        \end{center}

        \column{0.54\textwidth}
        \uncover<1->{\large\textbf{Confused?}}

        \vspace{0.5cm}

        \uncover<2->{
            If this felt like \textcolor{accent}{\textbf{a lot}} at once -\\
            that's because \textcolor{badred}{\textbf{it is !}}\\[0.4cm]
        }

        \vspace{0.6cm}

        \uncover<3->{
            \LARGE\textbf{We know deletion is complex - and that's \textcolor{accent}{\textit{okay!}}}
        }
    \end{columns}
\end{frame}

% % ── If You're Interested ─────────────────────────────────────
% \begin{frame}{If You're Interested\ldots}
%     \vspace{0.5cm}
%     \begin{center}
%         \large\textcolor{goodgreen}{\enspace\textbf{Want to go deeper?}}
%     \end{center}

%     \vspace{0.5cm}

%     \begin{itemize}
%         \item<2-> \large \textbf{CLRS Chapter 13} — full pseudocode \& proofs
%         \item<3-> \large \textbf{visualgo.net} — interactive online visualizer
%         \item<4-> \large \textbf{GitHub} — open-source implementations in your favourite language
%     \end{itemize}

%     \vspace{0.7cm}
%     \uncover<5->{
%     \begin{center}
%         \Large\textit{\enspace Focus on the \textbf{concepts}, not memorising every case!}
%     \end{center}
%     }
% \end{frame}

% ── Connector: Into Complexity ───────────────────────────────
\begin{frame}{Was All That Worth It?}
    \vspace{1.8cm}
    \begin{center}
        \LARGE\textbf{All that work.}\\[0.5cm]
        \uncover<2->{
            \Large\textcolor{accent}{What did it actually buy us?}
        }
    \end{center}

    \vspace{0.8cm}

    \uncover<3->{
    \begin{center}
        \large\textcolor{rbblack!70}{\textit{Let's finally see the payoff.}}
    \end{center}
    }
\end{frame}
