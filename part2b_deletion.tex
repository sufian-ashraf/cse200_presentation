% ============================================================
% part2b_deletion.tex  –  Red-Black Tree: Deletion
% ============================================================

% ── Deletion intro ─────────────────────────────────────────
\begin{frame}{Deletion: Even More Fun!}
    \begin{center}
        \Large\textbf{Deletion is even more\ldots\ interesting!} \faIcon{smile}
    \end{center}

    \pause

    \begin{columns}
        \column{0.5\textwidth}
        \begin{center}
            \textbf{Deleting a \textcolor{rbred}{RED} node}
        \end{center}
        \begin{itemize}
            \item No problem!
            \item Just remove it
            \item Properties still hold
        \end{itemize}

        \column{0.5\textwidth}
        \pause
        \begin{center}
            \textbf{Deleting a \textbf{BLACK} node}
        \end{center}
        \begin{itemize}
            \item Oh boy\ldots
            \item Black height changes!
            \item Need ``double black'' fix
            \item Complex cases
        \end{itemize}
    \end{columns}

    \pause
    \vspace{0.4cm}
    \begin{center}
        \Large\textbf{Let's see both cases\ldots}
    \end{center}
\end{frame}

% ── Deletion Decision Flowchart (HORIZONTAL LAYOUT) ──────────
\begin{frame}{Deletion Decision Flowchart}
    \vspace{0.1cm}
    \begin{center}
    \begin{tikzpicture}[scale=0.8, every node/.style={transform shape},
        node distance = 0.6cm and 1.2cm,
        decision/.style = {diamond, draw=blue!60, fill=blue!12,
                           text width=3.5em, text centered,
                           inner sep=0pt, minimum height=2.3em,
                           font=\scriptsize},
        action/.style   = {rectangle, draw=green!60!black, fill=green!12,
                           text width=4.5em, text centered,
                           rounded corners=3pt, minimum height=1.8em,
                           font=\scriptsize},
        problem/.style  = {rectangle, draw=red!60, fill=red!12,
                           text width=4.5em, text centered,
                           rounded corners=3pt, minimum height=1.8em,
                           font=\scriptsize},
        arrow/.style    = {thick,->,>=stealth}]

        % Start node
        \node[decision] (del) {Delete\\Node};
        \node[decision, right=1.5cm of del] (isred) {Is node\\RED?};

        % TOP branch (YES - RED node - easy path)
        \node[action, above right=0.3cm and 1.2cm of isred] (remove) {Simply\\Remove};
        \node[action, right=0.8cm of remove] (done1) {\faIcon{check}\\Done!};

        % BOTTOM branch (NO - BLACK node - hard path)
        \node[problem, below right=0.3cm and 1.2cm of isred] (black) {BLACK\\Node};
        \node[decision, right=0.8cm of black] (sibling) {Check\\Sibling};
        \node[problem, right=0.8cm of sibling] (cases) {Apply\\4 Cases};
        \node[action, right=0.8cm of cases] (done2) {Rebalance\\\faIcon{check}};

        % Arrows
        \draw[arrow] (del) -- (isred);
        \draw[arrow] (isred) |- node[above,near start,font=\tiny]{Yes} (remove);
        \draw[arrow] (remove) -- (done1);
        \draw[arrow] (isred) |- node[below,near start,font=\tiny]{No} (black);
        \draw[arrow] (black) -- (sibling);
        \draw[arrow] (sibling) -- (cases);
        \draw[arrow] (cases) -- (done2);
    \end{tikzpicture}
    \end{center}

    \vspace{0.3cm}
    \begin{center}
        \textcolor{green!60!black}{\textbf{Top path (RED)}} = easy \quad
        \textcolor{red!60}{\textbf{Bottom path (BLACK)}} = complex
    \end{center}
\end{frame}

% ── Easy case: Deleting RED node ───────────────────────────
\begin{frame}{Case 1: Deleting a \textcolor{rbred}{RED} Node (Easy!)}
    \begin{center}
        \Large Delete \textbf{5} from our tree
    \end{center}

    \vspace{0.3cm}

    \begin{columns}
        \column{0.5\textwidth}
        \begin{center}
            \textbf{Before}\\[0.25cm]
            \begin{tikzpicture}[scale=0.8]
                \node[blacknode] {2}
                    child {node[blacknode] {1}}
                    child {node[blacknode] {4}
                        child {node[rednode] {3}}
                        child {node[rednode] {5}}
                    };
            \end{tikzpicture}
        \end{center}

        \column{0.5\textwidth}
        \pause
        \begin{center}
            \textbf{After}\\[0.25cm]
            \begin{tikzpicture}[scale=0.8]
                \node[blacknode] {2}
                    child {node[blacknode] {1}}
                    child {node[blacknode] {4}
                        child {node[rednode] {3}}
                        child[missing] {}
                    };
            \end{tikzpicture}
        \end{center}
    \end{columns}

    \vspace{0.3cm}
    \pause

    \begin{center}
        \textbf{Why It's Easy}
    \end{center}
    \begin{itemize}
        \item Node 5 is \textcolor{rbred}{RED} and a leaf
        \item Simply remove it — black-height unchanged on all paths
        \item All properties still satisfied!
    \end{itemize}

    \pause
    \begin{center}
        \Large\textcolor{green!60!black}{\faIcon{check}}\; Done! That was nice!
    \end{center}
\end{frame}

% ── Hard case: Deleting BLACK node ─────────────────────────
\begin{frame}{Case 2: Deleting a \textbf{BLACK} Node (Uh oh\ldots)}
    \begin{center}
        \Large Delete \textbf{1} from our tree
    \end{center}

    \vspace{0.2cm}

    \begin{columns}
        \column{0.33\textwidth}
        \begin{center}
            \textbf{Before}\\[0.15cm]
            \begin{tikzpicture}[scale=0.65]
                \node[blacknode] (n2) {2}
                    child {node[blacknode] (n1) {1}}
                    child {node[blacknode] (n4) {4}
                        child {node[rednode] {3}}
                        child[missing] {}
                    };
                \draw[red, very thick, dashed] (n1) circle (0.55cm);
            \end{tikzpicture}
        \end{center}

        \column{0.33\textwidth}
        \pause
        \begin{center}
            \textbf{After Delete}\\[0.15cm]
            \begin{tikzpicture}[scale=0.65, every node/.style={transform shape}]
                \node[blacknode] (root) {2}
                    child {node[draw=none,fill=none,text=gray!50] (left) {X}} % X is the double-black placeholder
                    child {node[blacknode] (right) {4}
                        child {node[rednode] (r3) {3}}
                        child[missing] {}
                    };
                
                % Stable Arrow for Short Path (Left)
                \draw[->,thick,red] (left.south) -- ++(0,-0.6) 
                    node[below, font=\tiny\bfseries] {Short};
                    
                % Stable Arrow for Long Path (Right)
                \draw[->,thick,green!50!black] (r3.south) -- ++(0,-0.6) 
                    node[below, font=\tiny\bfseries]{Long};
            \end{tikzpicture}\\[0.1cm]
            \fcolorbox{red!80!black}{red!15}{\textcolor{red!70!black}{\bfseries\footnotesize\faIcon{times}~IMBALANCED}}
        \end{center}

        \column{0.33\textwidth}
        \pause
        \begin{center}
            \textbf{After Fix}\\[0.15cm]
            \begin{tikzpicture}[scale=0.65, every node/.style={transform shape}]
                % The balanced result after rotation and recoloring
                \node[blacknode] (n3) {3}
                    child {node[blacknode] (n2) {2}}
                    child {node[blacknode] (n4) {4}};

                
            \end{tikzpicture}\\[0.1cm]
            
            % Status Box
            \fcolorbox{green!60!black}{green!15}{%
                \textcolor{green!50!black}{\bfseries\footnotesize\faIcon{check-circle}~BALANCED}%
            }
            
            % Specific actions taken
            \vspace{0.2cm}
            \begin{itemize}
                \item {\tiny \faIcon{sync} \textbf{Rotate:} Right at 4, then 2}
                \item {\tiny \faIcon{paint-brush} \textbf{Recolor:} 3 $\to$ Black}
            \end{itemize}
        \end{center}
    \end{columns}

    \vspace{0.2cm}
    \pause
\end{frame}


% ── Cases 1 & 2 ─────────────────────────────────────────────
% Unified warm-amber block style matching the crane/orange theme
\setbeamercolor{block title}{bg=orange!70!black, fg=white}
\setbeamercolor{block body}{bg=orange!8!white, fg=black}
\begin{frame}[shrink=15]{Black Node Deletion Cases 1 \& 2}

    \begin{center}
    {\scriptsize
        \textbf{P}=Parent \enspace
        \textbf{S}=Sibling \enspace
        \textbf{L/R}=S's children \enspace
        \tikz[baseline=-0.5ex]\node[circle,draw=rbblack,fill=rbblack!10,
            text=rbblack,dashed,font=\bfseries\scriptsize,minimum size=5mm,
            inner sep=0pt]{DB}; = Double-Black node
    }
    \end{center}

    \vspace{0.02cm}

    \begin{columns}[t, onlytextwidth]

        \column{0.49\textwidth}
        \begin{block}{\faIcon{exchange-alt}\; Case 1: Sibling is RED}
            \centering\vspace{0.02cm}
            \begin{tikzpicture}[scale=0.82, every node/.style={transform shape},
                level distance=9mm, sibling distance=10mm,
                every node/.style={font=\small\bfseries, transform shape}]
                \node[blacknode] {P}
                    child {node[circle,draw=rbblack!70,fill=rbblack!8,
                                text=rbblack,dashed,minimum size=7mm,
                                font=\bfseries\small] {DB}}
                    child {node[rednode,draw=rbred,ultra thick] {S}
                        child {node[blacknode] {L}}
                        child {node[blacknode] {R}}
                    };
            \end{tikzpicture}

            \vspace{0.03cm}
            {\footnotesize\faIcon{sync-alt}~\textbf{Fix:} Rotate P left,
             recolor S $\to$ Black, P $\to$ Red\\[0.02cm]
            \textit{Converts to Case 2, 3, or 4}}
        \end{block}

        \column{0.49\textwidth}
        \begin{block}{\faIcon{paint-brush}\; Case 2: Sibling \& Children all BLACK}
            \centering\vspace{0.02cm}
            \begin{tikzpicture}[scale=0.82, every node/.style={transform shape},
                level distance=9mm, sibling distance=10mm,
                every node/.style={font=\small\bfseries, transform shape}]
                \node[blacknode] {P}
                    child {node[circle,draw=rbblack!70,fill=rbblack!8,
                                text=rbblack,dashed,minimum size=7mm,
                                font=\bfseries\small] {DB}}
                    child {node[blacknode] {S}
                        child {node[blacknode] {L}}
                        child {node[blacknode] {R}}
                    };
            \end{tikzpicture}

            \vspace{0.03cm}
            {\footnotesize\faIcon{arrow-up}~\textbf{Fix:} Recolor S $\to$ Red,\\
             push Double-Black up to P\\[0.02cm]
            \textit{Repeat fix from P if P was Black}}
        \end{block}

    \end{columns}

    \vspace{0.03cm}
    \begin{center}
        {\small\textcolor{gray}{\textit{Case 1 always leads to Case 2, 3, or 4 after rotation}}}
    \end{center}
\end{frame}

% ── Cases 3 & 4 ─────────────────────────────────────────────
\setbeamercolor{block title}{bg=orange!70!black, fg=white}
\setbeamercolor{block body}{bg=orange!8!white, fg=black}
\begin{frame}[shrink=15]{Black Node Deletion Cases 3 \& 4}

    \begin{center}
    {\scriptsize
        \textbf{P}=Parent \enspace
        \textbf{S}=Sibling \enspace
        \textbf{L/R}=S's children \enspace
        \tikz[baseline=-0.5ex]\node[circle,draw=rbblack,fill=rbblack!10,
            text=rbblack,dashed,font=\bfseries\scriptsize,minimum size=5mm,
            inner sep=0pt]{DB}; = Double-Black node
    }
    \end{center}

    \vspace{0.02cm}

    \begin{columns}[t, onlytextwidth]

        \column{0.49\textwidth}
        \begin{block}{\faIcon{redo}\; Case 3: Sibling's \textcolor[RGB]{120,20,30}{\textbf{Left}} child is RED}
            \centering\vspace{0.02cm}
            \begin{tikzpicture}[scale=0.82, every node/.style={transform shape},
                level distance=9mm, sibling distance=10mm,
                every node/.style={font=\small\bfseries, transform shape}]
                \node[blacknode] {P}
                    child {node[circle,draw=rbblack!70,fill=rbblack!8,
                                text=rbblack,dashed,minimum size=7mm,
                                font=\bfseries\small] {DB}}
                    child {node[blacknode] {S}
                        child {node[rednode,ultra thick] {L}}
                        child {node[blacknode] {R}}
                    };
            \end{tikzpicture}

            \vspace{0.03cm}
            {\footnotesize\faIcon{redo}~\textbf{Fix:} Right-rotate at S,
             swap colors of S \& L\\[0.02cm]
            \textit{Transforms into Case 4}}
        \end{block}

        \column{0.49\textwidth}
        \begin{block}{\faIcon{undo}\; Case 4: Sibling's Right child is RED}
            \centering\vspace{0.02cm}
            \begin{tikzpicture}[scale=0.82, every node/.style={transform shape},
                level distance=9mm, sibling distance=10mm,
                every node/.style={font=\small\bfseries, transform shape}]
                \node[blacknode] {P}
                    child {node[circle,draw=rbblack!70,fill=rbblack!8,
                                text=rbblack,dashed,minimum size=7mm,
                                font=\bfseries\small] {DB}}
                    child {node[blacknode] {S}
                        child {node[blacknode] {L}}
                        child {node[rednode,ultra thick] {R}}
                    };
            \end{tikzpicture}

            \vspace{0.03cm}
            {\footnotesize\faIcon{undo}~\textbf{Fix:} Left-rotate at P,
             recolor R $\to$ Black\\[0.02cm]
            \textbf{\textcolor{green!60!black}{\faIcon{check} Double-Black resolved!}}}
        \end{block}

    \end{columns}

    \vspace{0.03cm}
    \begin{center}
        {\textcolor{orange!70!black}{\faIcon{lightbulb}\;
        \textbf{Goal:} Always eventually reach Case 4 to fully eliminate Double-Black}}
    \end{center}
\end{frame}


% ── Confused Meme Slide ─────────────────────────────────────
\begin{frame}{Umm\ldots\ What?}
    \begin{columns}[c]
        \column{0.48\textwidth}
        \begin{center}
            \includegraphics[width=0.85\textwidth]{images/confused_cat.jpg}
        \end{center}

        \column{0.52\textwidth}
        \begin{center}
            {\Large\textbf{What just happened?}}

            \vspace{0.5cm}

            {\large\textit{``Four cases?\\[0.15cm]
             Rotations?\\[0.15cm]
             Recoloring?\\[0.3cm]
             Help!''}}

            \vspace{0.5cm}

            {\small\textcolor{gray}{Don't worry — even textbooks\\ span 20+ pages on this.}}
        \end{center}
    \end{columns}
\end{frame}

% ── Assurance Slide ─────────────────────────────────────────
\begin{frame}{Don't Worry!}
    \begin{center}
        {\normalsize\textbf{We know deletion is complex — and that's\, \textcolor{orange!80!red}{\textit{okay}!}}}
    \end{center}

    \vspace{0.1cm}

    \begin{alertblock}{\faIcon{exclamation-triangle}\; The Reality}
        \begin{itemize}\setlength\itemsep{0pt}
            \item The deletion algorithm is \textbf{huge} with many edge cases
            \item Each of the 4 cases has intricate implementation details
            \item Full implementation can span \textbf{hundreds} of lines
        \end{itemize}
    \end{alertblock}

    \vspace{0.08cm}

    \begin{center}
        \normalsize\textcolor{blue!80}{\faIcon{forward}\;\textbf{We'll skip the gory details for now!}}
    \end{center}

    \vspace{0.02cm}

    \begin{exampleblock}{\faIcon{lightbulb}\; If You're Interested\ldots}
        \begin{itemize}\setlength\itemsep{0pt}
            \item CLRS Chapter 13 — full pseudocode \& proofs
            \item Online visualizer: \texttt{visualgo.net}
            \item GitHub implementations in your favourite language
        \end{itemize}
    \end{exampleblock}

    \vspace{0.06cm}

    \begin{center}
        {\small\textit{\faIcon{brain}\; Focus on the \textbf{concepts}, not memorising every case!}}
    \end{center}
\end{frame}